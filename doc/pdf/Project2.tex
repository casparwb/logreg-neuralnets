%%
%% Automatically generated file from DocOnce source
%% (https://github.com/hplgit/doconce/)
%%
%%


%-------------------- begin preamble ----------------------

\documentclass[%
oneside,                 % oneside: electronic viewing, twoside: printing
final,                   % draft: marks overfull hboxes, figures with paths
10pt]{article}

\listfiles               %  print all files needed to compile this document

\usepackage{relsize,makeidx,color,setspace,amsmath,amsfonts,amssymb}
\usepackage[table]{xcolor}
\usepackage{bm,ltablex,microtype}

\usepackage[pdftex]{graphicx}

\usepackage[T1]{fontenc}
%\usepackage[latin1]{inputenc}
\usepackage{ucs}
\usepackage[utf8x]{inputenc}

\usepackage{lmodern}         % Latin Modern fonts derived from Computer Modern

% Hyperlinks in PDF:
\definecolor{linkcolor}{rgb}{0,0,0.4}
\usepackage{hyperref}
\hypersetup{
    breaklinks=true,
    colorlinks=true,
    linkcolor=linkcolor,
    urlcolor=linkcolor,
    citecolor=black,
    filecolor=black,
    %filecolor=blue,
    pdfmenubar=true,
    pdftoolbar=true,
    bookmarksdepth=3   % Uncomment (and tweak) for PDF bookmarks with more levels than the TOC
    }
%\hyperbaseurl{}   % hyperlinks are relative to this root

\setcounter{tocdepth}{2}  % levels in table of contents

% --- fancyhdr package for fancy headers ---
\usepackage{fancyhdr}
\fancyhf{} % sets both header and footer to nothing
\renewcommand{\headrulewidth}{0pt}
\fancyfoot[LE,RO]{\thepage}
% Ensure copyright on titlepage (article style) and chapter pages (book style)
\fancypagestyle{plain}{
  \fancyhf{}
  \fancyfoot[C]{{\footnotesize \copyright\ 1999-2019, "Data Analysis and Machine Learning FYS-STK3155/FYS4155":"http://www.uio.no/studier/emner/matnat/fys/FYS3155/index-eng.html". Released under CC Attribution-NonCommercial 4.0 license}}
%  \renewcommand{\footrulewidth}{0mm}
  \renewcommand{\headrulewidth}{0mm}
}
% Ensure copyright on titlepages with \thispagestyle{empty}
\fancypagestyle{empty}{
  \fancyhf{}
  \fancyfoot[C]{{\footnotesize \copyright\ 1999-2019, "Data Analysis and Machine Learning FYS-STK3155/FYS4155":"http://www.uio.no/studier/emner/matnat/fys/FYS3155/index-eng.html". Released under CC Attribution-NonCommercial 4.0 license}}
  \renewcommand{\footrulewidth}{0mm}
  \renewcommand{\headrulewidth}{0mm}
}

\pagestyle{fancy}


% prevent orhpans and widows
\clubpenalty = 10000
\widowpenalty = 10000

% --- end of standard preamble for documents ---


% insert custom LaTeX commands...

\raggedbottom
\makeindex
\usepackage[totoc]{idxlayout}   % for index in the toc
\usepackage[nottoc]{tocbibind}  % for references/bibliography in the toc

%-------------------- end preamble ----------------------

\begin{document}

% matching end for #ifdef PREAMBLE

\newcommand{\exercisesection}[1]{\subsection*{#1}}


% ------------------- main content ----------------------



% ----------------- title -------------------------

\thispagestyle{empty}

\begin{center}
{\LARGE\bf
\begin{spacing}{1.25}
Project 2 on Machine Learning, deadline November 8
\end{spacing}
}
\end{center}

% ----------------- author(s) -------------------------

\begin{center}
{\bf \href{{http://www.uio.no/studier/emner/matnat/fys/FYS3155/index-eng.html}}{Data Analysis and Machine Learning FYS-STK3155/FYS4155}}
\end{center}

    \begin{center}
% List of all institutions:
\centerline{{\small Department of Physics, University of Oslo, Norway}}
\end{center}
    
% ----------------- end author(s) -------------------------

% --- begin date ---
\begin{center}
Oct 2, 2019
\end{center}
% --- end date ---

\vspace{1cm}


\subsection*{Classification and Regression, from linear and logistic regression to neural networks}

The main aim of this project is to study both classification and
regression problems, where we can reuse the regression algorithms studied
in project 1. We will include logistic regresion for classification
problems and write our own multilayer perceptron code for studying
both regression and classification problems.  The codes developed in
project 1, including bootstrap and/or cross-validation as well as the
computation of the mean-squared error and/or the $R2$ or the accuracy score (classification problems) functions can
also be utilized in the present analysis. 


The data sets that we propose here are (the default sets)

\begin{itemize}
\item Regression (fitting a continuous function). In this part you will need to bring up your results from project 1 and compare these with what you get from you Neural Network code to be developed here. The data sets could be
\begin{enumerate}

 \item Either the Franke function or the terrain data from project 1, or data sets your propose.

\end{enumerate}

\noindent
\item Classification. Here you will also need to develop a Logistic regression code that you will use to compare with the Neural Network code. The data set we propose are
\begin{enumerate}

 \item The credit card data set from \href{{https://archive.ics.uci.edu/ml/datasets/default+of+credit+card+clients}}{UCI}. This data set links to a \href{{https://bradzzz.gitbooks.io/ga-seattle-dsi/content/dsi/dsi_05_classification_databases/2.1-lesson/assets/datasets/DefaultCreditCardClients_yeh_2009.pdf}}{recent scientific article}.  Furthermore, in the lecture slides on \href{{https://compphysics.github.io/MachineLearning/doc/pub/LogReg/html/LogReg.html}}{Logistic Regression}, you will find an example code for the Credit Card data. You could use this code as an example on how to read the data and use \textbf{Scikit-Learn} to run a classification problem.  
\end{enumerate}

\noindent
\end{itemize}

\noindent
However, if you would like to study other data sets, feel free to
propose other sets. What we listed here are mere suggestions from our
side. If you opt for another data set, consider using a set which
has been studied in the scientific literature. This makes it easier
for you to compare and analyze your results. It is also an essential
elements of the scientific discussion.

\paragraph{Part a): Write your Logistic Regression code, first step.}
If you opt for the credit card data, your first task is to familiarize yourself with the data set and the scientific article. 
We recommend also that you study the code example in the \href{{https://compphysics.github.io/MachineLearning/doc/pub/LogReg/html/LogReg.html}}{Logistic Regression}. 

Write the part of the code which reads in the data and sets up the relevant data sets. 

\paragraph{Part b): Write your Logistic Regression code, second step.}
Now you should write your Logistic Regression code with the aim to
reproduce the Logistic Regression analysis of the \href{{https://bradzzz.gitbooks.io/ga-seattle-dsi/content/dsi/dsi_05_classification_databases/2.1-lesson/assets/datasets/DefaultCreditCardClients_yeh_2009.pdf}}{scientific
article}.

Define your cost function and the design matrix before you start writing your code.

In order to find the optimal parameters of your logistic regressor you
should include a gradient descent solver, as discussed in the
\href{{https://compphysics.github.io/MachineLearning/doc/pub/Splines/html/Splines-bs.html}}{gradient descent
lectures}.
Since we don't have so many data points, you may just code the
standard gradient descent with a given learning rate, or even attempt
to use the Newton-Raphson method.  Alternatively, it may be useful for
the next part on neural networks to implement a stochastic gradient
descent with and without mini-batches. Stochastic gradient with
mini-batches may give the best results. You could finally compare your
code with the output from \textbf{scikit-learn}'s toolbox for optimization
methods applied to logistic regression.


To measure the performance of our classification problem we use the
so-called \emph{accuracy} score.  The accuracy is as you would expect just
the number of correctly guessed targets $t_i$ divided by the total
number of targets. A perfect classifier will have an accuracy score of
$1$.

\[ 
\text{Accuracy} = \frac{\sum_{i=1}^n I(t_i = y_i)}{n} ,
\]

where $I$ is the indicator function, $1$ if $t_i = y_i$ and $0$
otherwise if we have a binary classifcation problem. Here $t_i$
represents the target and $y_i$ the outputs of your Logistic
Regression code.


You can compare your own results with those obtained using
\textbf{scikit-learn}.


\paragraph{Part c): Writing your own Neural Network code.}
Your aim now, and this is the central part of this project, is to
write to your own Feed Forward Neural Network  code implementing the back
propagation algorithm discussed in the \href{{https://compphysics.github.io/MachineLearning/doc/pub/NeuralNet/html/NeuralNet-bs.html}}{lecture
slides}. We
start with the Logistic Regression  case and the data set discussed in parts a) and b) but train
now the network to find the optimal weights and biases. You are free
to use the codes in the above lecture slides as starting points.

Discuss again your choice of cost function.

Train your network and compare the results with those from your Logistic  Regression code. 
You can test your results against a similar code using \textbf{Scikit-Learn} (see the examples in the above lecture notes) or \textbf{tensorflow/keras}. 

Comment your results and give a critical discussion of the results
obtained with the Logistic Regression code and your own Neural Network
code.  Make an analysis of the regularization parameters and the learning rates employed to find the optimal accurary score.

A useful reference on the back progagation algorithm is \href{{http://neuralnetworksanddeeplearning.com/}}{Nielsen's
book}. It is an excellent
read.


\paragraph{Part d): Regression analysis using neural networks.}
Here we will change the cost function for our neural network code
developed in part c) in order to perform a regression (fitting a
function or some data set) analysis. As stated above, our default data
sets could be either the Franke function or the terrain data from
project 1.

Compare you results from the neural network regression analysis (with a discussion of learning rates and regularization parameters)
with those you obtained in project 1. Alternatively, if you opt for other data sets, you would need to run your standard ordinary least squares, Ridge and Lasso calculations using your codes from project 1.

\paragraph{Part e) Critical evaluation of the various algorithms.}
After all these glorious calculations, you should now summarize the various algorithms and come with a critical evaluation of their pros and cons. Which algorithm works best for the regression case and which is best for the classification case. These codes can also be part of your final project 3, but now applied to other data sets.




\subsection*{Background literature}

\begin{enumerate}
\item The text of Michael Nielsen is highly recommended, see \href{{http://neuralnetworksanddeeplearning.com/}}{Nielsen's book}. It is an excellent read.

\item The textbook of \href{{https://www.springer.com/gp/book/9780387848570}}{Trevor Hastie, Robert Tibshirani, Jerome H. Friedman, The Elements of Statistical Learning, Springer}, chapters 3 and 7 are the most relevant ones for the analysis here. 

\item \href{{https://arxiv.org/abs/1803.08823}}{Mehta et al, arXiv 1803.08823}, \emph{A high-bias, low-variance introduction to Machine Learning for physicists}, ArXiv:1803.08823.
\end{enumerate}

\noindent
\subsection*{Introduction to numerical projects}

Here follows a brief recipe and recommendation on how to write a report for each
project.

\begin{itemize}
  \item Give a short description of the nature of the problem and the eventual  numerical methods you have used.

  \item Describe the algorithm you have used and/or developed. Here you may find it convenient to use pseudocoding. In many cases you can describe the algorithm in the program itself.

  \item Include the source code of your program. Comment your program properly.

  \item If possible, try to find analytic solutions, or known limits in order to test your program when developing the code.

  \item Include your results either in figure form or in a table. Remember to        label your results. All tables and figures should have relevant captions        and labels on the axes.

  \item Try to evaluate the reliabilty and numerical stability/precision of your results. If possible, include a qualitative and/or quantitative discussion of the numerical stability, eventual loss of precision etc.

  \item Try to give an interpretation of you results in your answers to  the problems.

  \item Critique: if possible include your comments and reflections about the  exercise, whether you felt you learnt something, ideas for improvements and  other thoughts you've made when solving the exercise. We wish to keep this course at the interactive level and your comments can help us improve it.

  \item Try to establish a practice where you log your work at the  computerlab. You may find such a logbook very handy at later stages in your work, especially when you don't properly remember  what a previous test version  of your program did. Here you could also record  the time spent on solving the exercise, various algorithms you may have tested or other topics which you feel worthy of mentioning.
\end{itemize}

\noindent
\subsection*{Format for electronic delivery of report and programs}

The preferred format for the report is a PDF file. You can also use DOC or postscript formats or as an ipython notebook file.  As programming language we prefer that you choose between C/C++, Fortran2008 or Python. The following prescription should be followed when preparing the report:

\begin{itemize}
  \item Use Devilry to hand in your projects, log in  at  \href{{http://devilry.ifi.uio.no}}{\nolinkurl{http://devilry.ifi.uio.no}} with your normal UiO username and password and choose either 'fysstk3155' or 'fysstk4155'. There you can load up the files within the deadline.

  \item Upload \textbf{only} the report file!  For the source code file(s) you have developed please provide us with your link to your github domain.  The report file should include all of your discussions and a list of the codes you have developed.  Do not include library files which are available at the course homepage, unless you have made specific changes to them.

  \item In your git repository, please include a folder which contains selected results. These can be in the form of output from your code for a selected set of runs and input parameters.

  \item In this and all later projects, you should include tests (for example unit tests) of your code(s).

  \item Comments  from us on your projects, approval or not, corrections to be made  etc can be found under your Devilry domain and are only visible to you and the teachers of the course.
\end{itemize}

\noindent
Finally, 
we encourage you to collaborate. Optimal working groups consist of 
2-3 students. You can then hand in a common report. 



\subsection*{Software and needed installations}

If you have Python installed (we recommend Python3) and you feel pretty familiar with installing different packages, 
we recommend that you install the following Python packages via \textbf{pip} as
\begin{enumerate}
\item pip install numpy scipy matplotlib ipython scikit-learn tensorflow sympy pandas pillow
\end{enumerate}

\noindent
For Python3, replace \textbf{pip} with \textbf{pip3}.

See below for a discussion of \textbf{tensorflow} and \textbf{scikit-learn}. 

For OSX users we recommend also, after having installed Xcode, to install \textbf{brew}. Brew allows 
for a seamless installation of additional software via for example
\begin{enumerate}
\item brew install python3
\end{enumerate}

\noindent
For Linux users, with its variety of distributions like for example the widely popular Ubuntu distribution
you can use \textbf{pip} as well and simply install Python as 
\begin{enumerate}
\item sudo apt-get install python3  (or python for python2.7)
\end{enumerate}

\noindent
etc etc. 

If you don't want to install various Python packages with their dependencies separately, we recommend two widely used distrubutions which set up  all relevant dependencies for Python, namely
\begin{enumerate}
\item \href{{https://docs.anaconda.com/}}{Anaconda} Anaconda is an open source distribution of the Python and R programming languages for large-scale data processing, predictive analytics, and scientific computing, that aims to simplify package management and deployment. Package versions are managed by the package management system \textbf{conda}

\item \href{{https://www.enthought.com/product/canopy/}}{Enthought canopy}  is a Python distribution for scientific and analytic computing distribution and analysis environment, available for free and under a commercial license.
\end{enumerate}

\noindent
Popular software packages written in Python for ML are

\begin{itemize}
\item \href{{http://scikit-learn.org/stable/}}{Scikit-learn}, 

\item \href{{https://www.tensorflow.org/}}{Tensorflow},

\item \href{{http://pytorch.org/}}{PyTorch} and 

\item \href{{https://keras.io/}}{Keras}.
\end{itemize}

\noindent
These are all freely available at their respective GitHub sites. They 
encompass communities of developers in the thousands or more. And the number
of code developers and contributors keeps increasing.
















\begin{enumerate}
\item For project 3, you should feel free to use your own codes from projects 1 and 2, eventually write your own for SVMs and/or Decision trees and random forests' or use the available functionality of \textbf{scikit-learn}, \textbf{tensorflow}, etc. 

\item The estimates you used and tested in projects 1 and 2 should also be included, that is the $R2$-score, \textbf{MSE}, cross-validation and/or bootstrap if these are relevant. 

\item If possible, you should link the data sets with exisiting research and analyses thereof. Scientific articles which have used Machine Learning algorithms to analyze the data are highly welcome. Perhaps you can improve previous analyses and even publish a new article? 

\item A critical assessment of the methods with ditto perspectives and recommendations is also something you need to include.
\end{enumerate}

\noindent

% ------------------- end of main content ---------------

\end{document}

